\section{System Architecture}

The \textbf{RoadSense} consists of multiple IoT devices installed in vehicles, communicating with a central server designed to be highly scalable to handle data from thousands of devices. In this section we will describe all meaningful components of the system, focusing on the IoT data pipeline and the client-server architecture.

\subsection{IoT Data Pipeline}
\label{subsec:iot_data_pipeline}

The data pipeline has been designed with scalability in mind, allowing for efficient data collection, processing, and data analysis from multiple (potentially thousands) concurrent IoT devices. The following diagram illustrates the pipeline steps, from data ingestion to the storage of processed data.

\begin{figure}[H]
	\centering
	\includegraphics[width=\textwidth]{../../assets/diagrams/iot_data_pipeline/iot_data_pipeline.png}
	\caption{IoT Data Pipeline}
\end{figure}

The pipeline consists of the following components:

\begin{enumerate}
	\item \textbf{Data ingestion}: IoT devices collect vibration, GPS, and other relevant data points. When the vehicle reaches an access point, the data will be transmitted to the server. Each device will be able to connect to different Wi-Fi networks, allowing for data transmission in different locations. This may include public Wi-Fi networks, cellular data, or a dedicated network infrastructure.

	\item \textbf{Authentication and security}: In the original idea of the project each device is authenticated before data transmission to ensure data integrity and prevent unauthorized access. For this purpose, was decided to use \href{https://www.keycloak.org/}{Keycloak} for identity and access management. Unfortunately, this feature was not implemented in the presented prototype in order to focus on the core functionality of the system.

	\item \textbf{Geographical distribution}: The server leverages DNS-based load balancing to distribute incoming data across regional gateways for efficient processing. We have chosen to use \href{https://www.isc.org/bind/}{BIND9} for DNS-based load balancing along with \href{https://www.maxmind.com/en/geoip2-databases}{GeoIP} for geolocation. Each regional gateway will be responsible for routing the user requests to the regional message queuing system. For this purpose, we will use \href{https://traefik.io/}{Traefik} as the reverse proxy. Unfortunately, also this feature was not implemented in the prototype as it would introduce additional complexity to the system. However, this feature is essential for the scalability of the system as it allows for efficient data processing across multiple regions.

	\item \textbf{Message queues}: Each gateway node processes incoming data and forwards it to a regional queuing system to allow for parallel processing. After evaluating multiple options, we decided to use \href{https://www.rabbitmq.com/}{RabbitMQ} as the message queue system. To ensure high availability, we will deploy RabbitMQ in a cluster configuration (refer to the \href{https://www.rabbitmq.com/clustering.html}{RabbitMQ Clustering Guide}). For the prototype we avoided the creation of a RabbitMQ cluster, and was used a single instance of RabbitMQ. This feature is still of interest for the scalability of the system.

	\item \textbf{Data Consumers and Preprocessing}: Each region is served by a set of consumer microservices that retrieve incoming data from the RabbitMQ clusters, perform data validation, and execute map matching to process and align the data with geographical coordinates.

	      During the initial project specifications, \href{https://golang.org/}{Go} was selected as the primary language for these microservices. However, the prototype implementation utilized \href{https://rust-lang.org/}{Rust} instead. This decision was driven by the opportunity to explore Rust's performance and safety features. The microservices were designed to be lightweight, efficient, and resilient against failures.

	\item \textbf{Data storage}: Processed data is stored in a scalable database system that can handle high volumes of data. Since we are dealing with both date-time and geospatial data, we chose to use \href{https://www.timescale.com/}{TimescaleDB} as the database system with the \href{https://postgis.net/}{PostGIS} extension to support geospatial queries.
	      PostGIS was used to store and query collected samples
\end{enumerate}

\noindent \textit{Note:} Although not implemented in the current prototype, most of the services/microservices described above could be containerized (if not already) using technologies like Docker and managed in a production environment using \href{https://kubernetes.io/}{Kubernetes} for scalability and orchestration.

\subsection{Client-Server interaction}

The \textbf{RoadSense} system employs a client-server architecture designed to efficiently deliver real-time road condition data. The client is a web-based application that visualizes road condition samples on an interactive map. These samples are fetched from a custom API microservice, which only returns data points within the map's current bounding box, leveraging \textbf{PostGIS} for spatial queries to optimize performance and minimize data transfer. This ensures scalability and efficient handling of large datasets, as the system can support millions of samples without overwhelming either the client or the server. The following diagram illustrates the client-server interaction:

\begin{figure}[H]
	\centering
	\includegraphics[width=\textwidth]{../../assets/diagrams/client_server_architecture/client_server_arch.png}
	\caption{Client-Server Interaction}
\end{figure}

\noindent The backend is implemented as a custom API server built with \textbf{Rust} using the \textbf{Actix} web framework and \href{https://diesel.rs/}{Diesel} for database interaction. It provides a read-only interface, returning road condition samples in \texttt{JSON} format based on client requests. Data insertion is handled by separate consumers processing messages from \textbf{RabbitMQ}. The architecture enables efficient data processing, retrieval, and real-time visualization while maintaining scalability and performance.
