The \textbf{RoadSense} project seeks to develop an IoT-powered system for detecting and mapping road anomalies, such as potholes and uneven surfaces. By equipping multiple vehicles with sensor nodes, the system will gather and analyze road vibration data to generate an interactive, detailed heatmap of road conditions. This data will be instrumental in optimizing road maintenance, enhancing driver safety, and providing real-time hazard alerts.

The project aims to deliver a comprehensive solution for road condition monitoring by addressing key objectives across data collection, processing, visualization, and alerting. Specifically, the system will focus on:

\begin{enumerate}[label=\arabic*.]
	\item Designing a cost-effective IoT-based solution for detecting and mapping road anomalies.
	\item Measuring road bumpiness and issuing real-time alerts for hazardous conditions.
	\item Providing a user-friendly interface for stakeholders to visualize road conditions and manage alerts effectively.
	\item Enhancing road condition accuracy through data collection from multiple vehicles.
\end{enumerate}

\noindent In the following sections, we will delve into the system design of \textbf{RoadSense} project and explore the design choices, components, and technologies employed to achieve these objectives.

\pagebreak
