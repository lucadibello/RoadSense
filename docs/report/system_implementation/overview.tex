This chapter details the implementation of the \textbf{RoadSense} system, focusing on the practical realization of its architecture and components. The system integrates various technologies to ensure reliable data collection, processing, and visualization for monitoring road conditions.

The implementation covers three main areas: the IoT sensor nodes deployed in vehicles for data acquisition, the backend infrastructure responsible for data aggregation and analysis, and the client-side application used for data visualization and user interaction. Each component is designed to optimize performance, scalability, and usability.

The IoT sensor nodes preprocess data locally to reduce transmission overhead while ensuring accurate representation of road states. The backend, built using a microservice-based approach, processes incoming data, stores it efficiently, and provides APIs for real-time access to relevant datasets. The client application uses these APIs to present road condition data interactively on a map, supporting features such as filtering, heatmaps, and severity-based color coding.

This chapter provides detailed insights into the implementation of these components, explaining the choices of technologies and methodologies employed to achieve the desired functionality and performance of the system.

\pagebreak
