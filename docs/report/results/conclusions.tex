\section{Conclusions}

The \textbf{RoadSense} project has been a valuable learning experience for the team. It allowed us to work with edge computing, IoT, and the challenges of building a system that combines both. For many of us, it was our first time working with embedded systems, making this project a great chance to learn new skills. We also gained a better understanding of the importance of data collection and how it can be used to make better decisions.

The project achieved its main goal of creating a system that monitors road conditions and provides real-time feedback. The system can map road conditions, detect potholes, and display the collected data through a web interface. This allows users to see road conditions in real-time and provides a solid base for future improvements.

Although the prototype works as expected, there are areas that can be improved, such as making the map-matching process more accurate, improving the frontend design, and implementing the full scalable pipeline planned during the design phase. These are good next steps for future versions of the system.

We would like to thank our supervisors for their guidance and support during the project. We are also grateful to the Università della Svizzera italiana (USI) for giving us the resources needed to complete this work. This project has been an important step in learning and growing our technical and teamwork skills.
