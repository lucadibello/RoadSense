\section{Known problems}

The current implementation of the \textbf{RoadSense} system, while functional, presents several limitations and areas for improvement:

\begin{itemize}
    \item \textbf{Map Matching:} The map-matching functionality relies on the OSMR service, which does not always provide optimal results. This can lead to inaccuracies in aligning road condition data with geographical locations.
    \item \textbf{Frontend Application:} The frontend application is minimalistic and lacks advanced features. Enhancements to the user interface and the addition of anticipated features, such as an automated management system for road condition data, are necessary to improve usability.
    \item \textbf{Unmet Objectives:} Certain objectives outlined during the planning phase, such as the geographical distribution of the system, were not fully achieved in the prototype.
    \item \textbf{Scalable Pipeline:} The current prototype does not implement the scalable pipeline envisioned during the specification phase. Instead, it focuses on a simplified version to validate the core functionality.
\end{itemize}

\noindent These limitations highlight the areas that need further development to achieve the full potential of the \textbf{RoadSense} system in future iterations.
